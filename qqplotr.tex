% !TeX root = RJwrapper.tex
\title{ggplot2 compatible Quantile-quantile plots in R}
\author{by Alexandre Almeida, Adam Loy, Heike Hofmann}

\maketitle

\abstract{%
An abstract of less than 150 words.
}

\newcommand{\hh}[1]{{\textcolor{orange}{#1}}}

\subsection{TODO:}\label{todo}

\begin{itemize}
\tightlist
\item
  Abstract
\item
  Package implementation
\item
  Give examples

  \begin{itemize}
  \tightlist
  \item
    Heike: BRFSS example
  \end{itemize}
\item
  Conclusion
\end{itemize}

\subsection{Introduction}\label{introduction}

\label{sec:introduction}

Univariate distributional assessment is a common thread throughout
statistical analyses during both the exploratory and confirmatory
stages. When we begin exploring a new data set we often consider the
distribution of individual variables before moving on to explore
multivariate relationships. After a model has been fit to a data set, we
must assess whether the distributional assumptions made are reasonable,
and if they are not, then we must understand the impact this has on the
conclusions of the model. Graphics provide arguably the most common way
to carry out these univariate assessments. While there are many plots
that can be used for distribution exploration and assessment, a
quantile-quantile (Q-Q) plot \citep{Wilk1968-ii} is one of the most
common plots used.

Q-Q plots compare two distributions by comparing a commmon set of
quantiles. To compare a sample, \(y_1, y_2, \ldots, y_n\) to a
theoretical distribution, a Q-Q plot is simply a scatterplot of the
sample quantiles, \(y_{(i)}\), against the corresponding quantiles from
the theoretical distribution, \(F^{-1}\left( F_n(y_i) \right)\). If the
empirical distribution is consistent with the theoretical distribution,
then the Q-Q plot will be linear. For example, Figure \ref{fig:ex-qq1}
shows two Q-Q plots: the left plot compares a sample drawn from the
lognormal distribution to the lognormal distribution, while the right
plot compares a sample drawn from the lognormal distribution to the
normal distribution. As expected, the lognormal Q-Q plot is
approximately linear as the data and model are in agreement, while the
normal Q-Q plot is curved, indicating disagreement between the data and
the model.

\begin{Schunk}
\begin{figure}

{\centering \includegraphics[width=.4\linewidth]{qqplotr_files/figure-latex/ex-qq1-1} \includegraphics[width=.4\linewidth]{qqplotr_files/figure-latex/ex-qq1-2} 

}

\caption[The left plot compares a sample drawn from the lognormal distribution to the lognormal distribution, while the right plot compares a sample drawn from the lognormal distribution to the normal distribution]{The left plot compares a sample drawn from the lognormal distribution to the lognormal distribution, while the right plot compares a sample drawn from the lognormal distribution to the normal distribution. The curvature in the normal Q-Q plot highlights the disagreement betweeen the data and the model.}\label{fig:ex-qq1}
\end{figure}
\end{Schunk}

Additional graphical elements are often added to Q-Q plots in order to
aid in distributional assessment including. A reference line is often
added to a Q-Q plot to assist the detection of departures from
normality. This line is often drawn by connecting the first and third
quartiles. Pointwise or simultaneous confidence bands are also
frequently built around the reference line to display the expected
degree of sampling error for the proposed model, so that minor
deviations from the reference line are not over-interpreted. Figure
\ref{fig:ex-qq2} adds such reference lines and 95\% pointwise confidence
bands to the Q-Q plots in Figure \ref{fig:ex-qq1}.

\begin{Schunk}
\begin{figure}

{\centering \includegraphics[width=.4\linewidth]{qqplotr_files/figure-latex/ex-qq2-1} \includegraphics[width=.4\linewidth]{qqplotr_files/figure-latex/ex-qq2-2} 

}

\caption[Adding reference lines and $95\%$ pointwise confidence bands to the Q-Q plots in Figure 1]{Adding reference lines and $95\%$ pointwise confidence bands to the Q-Q plots in Figure 1.}\label{fig:ex-qq2}
\end{figure}
\end{Schunk}

Different orientations of Q-Q plots have also been proposed, most
notably the de-trended Q-Q plot. To detrend a Q-Q plot, the \(y\)-axis
is changed to show the difference between the observed quantile and the
reference line. Consequently, the line representing the agreement with
the theoretical distribution is the \(x\)-axis. \citet{Loy2016-fg} find
that detrended Q-Q plots are more powerful than other designs, so long
at the \(y\)-axis limits are set so that the aspect ratio is kept the
same as in the traditional Q-Q plot, which they call \emph{adjusted
de-trended Q-Q plots}. Figure \ref{fig:ex-detrend} displays the normal
Q-Q plot from Figure \ref{fig:ex-qq2} along with a detrended version.

\begin{Schunk}
\begin{figure}

{\centering \includegraphics[width=.4\linewidth]{qqplotr_files/figure-latex/ex-detrend-1} \includegraphics[width=.4\linewidth]{qqplotr_files/figure-latex/ex-detrend-2} 

}

\caption[The left plot displays a traditional normal Q-Q plot for data simulated from a lognormal distributin]{The left plot displays a traditional normal Q-Q plot for data simulated from a lognormal distributin. The right plot displays an adjusted detrended Q-Q plot of the same data, created by plotting the differences between the sample quantiles and the proposed model on the $y$-axis.}\label{fig:ex-detrend}
\end{figure}
\end{Schunk}

Various implementations of Q-Q plots exist in R. Normal Q-Q plots, where
a sample is compared to the standard normal distribution, are
implemented using \texttt{qqplot} and \texttt{qqline} in \pkg{base}
graphics \citep{R}. \pkg{lattice} provides a general framework for Q-Q
plots in the \texttt{qqmath} function, allowing one to compare a sample
to any theoretical distribution by specifying the quantile function
\citep{lattice}. \texttt{qqPlot} in the \pkg{car} package also allows
for the assessment of non-normal distributions and adds pointwise
confidence bands via normal theory or the parametric bootstrap
\citep{car}. \pkg{ggplot2} provides \texttt{geom\_qq} and
\texttt{geom\_qq\_line}, enabling the creation of Q-Q plots with a
reference line, much like those created using \texttt{qqmath}
\citep{ggplot2}. None of these general use packages allow for easy
construction of de-trended Q-Q plots.

\pkg{qqplotr} extends \pkg{ggplot2} to provide a complete implementation
of Q-Q plots. The package allows for quick construction of all Q-Q plot
designs without sacrificng the flexibility of the \pkg{ggplot2}
framework. In the remainder of this paper, we will introduce the
plotting framework provided by \pkg{qqplotr} and provide multiple
examples of how it can be used.

\subsection{\texorpdfstring{Implementing probability plots in the
\pkg{ggplot2}
framework}{Implementing probability plots in the  framework}}\label{implementing-probability-plots-in-the-framework}

With \pkg{qqplotr} we extend some of the original \pkg{ggplot2} quantile
plot functionatilites by permitting the drawing of Q-Q points, lines,
and confidence bands. Our approach provides a \pkg{ggplot2} layering
mechanism so that for each one of those plot elements we implemented a
\pkg{ggplot2} ``stat'' (statistical transformation). In addition, we
also implemented a \pkg{ggplot2} ``geom'' (geometrical object)
specifically for the confidence bands. That geom permits a simpler way
of handling graphical parameters, which will become clearer in the
\nameref{sec:examples} section.

The Q-Q plot functions are divided into three statistical
transformations:

\begin{itemize}
\item
  \texttt{stat\_qq\_point}: a modified version of \texttt{stat\_qq} from
  \pkg{ggplot2} that plots the sample quantiles versus the theoretical
  quantiles (as in Figure \ref{fig:qq}). The novelty of this
  implementation is an option to detrend the plotted points (see
  \nameref{sec:introduction}). All other implemented functions in this
  package also allow the detrend adjustment.
\item
  \texttt{stat\_qq\_line}: draws a reference line based on the sample
  data quantiles, defaulting to the first and third quartiles.
\item
  \texttt{stat\_qq\_band}: draws confidence bands around the reference
  line using one of three methods: a normal approximation, the
  parametric bootstrap, or the tail-sensitive procedure.

  \begin{itemize}
  \tightlist
  \item
    \textbf{Normal:} Specifying \texttt{bandType\ =\ "norm"} constructs
    pointwise confidence bands based on the normal approximation to the
    distribution of the order statistics. For example, an approximate
    95\% confidence interval for the \(i\)th order statistic is
    \(\widehat{X}_{(i)}~\pm~\Phi^{-1}(.975)~\cdot~SE(X_{(i)})\), where
    \(\widehat{X}_{(i)}\) denotes the value along the fitted line,
    \(\Phi^{-1}(\cdot)\) denotes the quantile function for the standard
    normal distribution, and \(SE(X_{(i)})\) is the standard error of
    the \(i\)th order statistic.
  \item
    \textbf{Bootstrap:} Specifying \texttt{bandType\ =\ "bs"} constructs
    pointwise confidence bands using percentile confidence intervals
    from the parametric bootstrap.
  \item
    \textbf{Tail-sensitive:} Specifying \texttt{bandType\ =\ "ts"}
    constructs the simulation-based tail-sensitive simultaneous
    confidence bands proposed by \citet{Aldor-Noiman2013-xw}.
  \end{itemize}
\end{itemize}

\subsection{Examples}\label{examples}

\label{sec:examples}

In this section, we demonstrate the capabilities of the \pkg{qqplotr}
package. We start by loading the package:

\begin{Schunk}
\begin{Sinput}
# also loads ggplot2
library(qqplotr)
\end{Sinput}
\end{Schunk}

\subsubsection{BRFSS example}\label{brfss-example}

The Center for Disease Control and Prevention runs an annual telephone
survey, the Behavioral Risk Factor Surveillance System (BRFSS), to keep
track of the US populations' `health-related risk behaviors, chronic
health conditions, and use of preventive services'.

Close to half a million interviews are conducted each year. Here, we are
focussing on the 2012 responses for Iowa. 7166 responses were gathered
across 359 questions and derived variables. Among these, are people's
height and weight, which we are going to assess in more detail.

Figure \ref{fig:heights} shows two Q-Q plots side by side. For each of
the plots, a sample of 200 men and 200 women is drawn from the overall
number of responses. On the left hand side, individuals' heights are
plotted in a Q-Q plot comparing raw heights to a normal distribution. We
see that the distributions for both men and women (colour) is showing
horizontal steps: this indicates that the distributional assessement is
heavily dominated by the discreteness in the data, as most survey
participants responded to the question of their height to the closest
inch. On the right hand side of Figure \ref{fig:heights}, we use
jittering; this means that we add a random number generated from a
random uniform distribution on \(\pm 0.5\) inch to the reported height.
By this mean we diminish the effect that discreteness might have on the
distribution. This brings the observed distribution much closer to a
normal distribution. Note that separate normal distributions were fitted
for each gender, not surprisingly, the resulting distributions have
different means (women are on average 6 inch shorter than men in this
dataset). Interestingly, the slope of the two genders is similar,
indicating that the same scale parameter fits both genders'
distributions (the standard deviation of height in the data set is 2.97
inch for men and 2.91 inch for women, see Table \ref{tab:heights}). The
dark line between the two groups is the identity line - indicating the
theoretical distribution each of these groups are compared to. This
distribution is based on parameters estimated from the whole population
(see Table \ref{tab:heights} for numbers) . While the mean is about half
way between the gender means, we see from the higher slope of the line
that in comparison to each group, the standard deviation of the height
based on the whole population is larger.

\begin{table}

\caption{\label{tab:heights-table}Summary of Iowa's residents height and standard deviation (in inch) by gender and total.\label{tab:heights}}
\centering
\begin{tabular}[t]{lrr}
\toprule
SEX & mean & sd\\
\midrule
Male & 70.55 & 2.97\\
Female & 64.51 & 2.91\\
Total & 66.99 & 4.18\\
\bottomrule
\end{tabular}
\end{table}

\begin{Schunk}
\begin{figure}

{\centering \includegraphics[width=\textwidth]{qqplotr_files/figure-latex/heights-1} 

}

\caption[Sample (200 men and 200 women) of raw heights (left) and jittered heights (right)]{Sample (200 men and 200 women) of raw heights (left) and jittered heights (right). The distribution on the left is dominated by the discreteness of the data. On the right we see that except for some outliers an assumption of normality for people's height is not completely absurd.}\label{fig:heights}
\end{figure}
\end{Schunk}

Unlike respondents' heights, their weigths do not seem to be normally
distributed. Figure \ref{fig:weights} shows again two Q-Q plots. The Q-Q
plot on the left uses raw weights and compares to a normal distribution.
From the curved points we see that tails of the observed distribution
are heavier than expected under a normal distribution. On the right,
weights are log-transformed. We see that a normal distribution for each
of the genders shows --with the exceptions of a few extreme outliers-- a
reasonable fit.

\begin{Schunk}
\begin{figure}

{\centering \includegraphics[width=\textwidth]{qqplotr_files/figure-latex/weights-1} 

}

\caption[Sample (200 men and 200 women) of weights]{Sample (200 men and 200 women) of weights. Unlike people's height, weight seems to be heavily right skewed with some additional outliers on the extreme left (left plot). On the right, weight was log-transfomed before its distribution is compared to a theoretical normal. }\label{fig:weights}
\end{figure}
\end{Schunk}

Instead of transforming the observed values, we can change the
theoretical distribution against which we compare. Figure
\ref{fig:weights-log} shows two Q-Q plots where a log-normal
distribution is chosen as the theoretical distribution. On the left, we
compare against a log-normal distribution with mean 4.389 and standard
deviation 0.223 (the log-transformed averages of average weight and
standard deviation in Iowa's population). Again, the fits seem
reasonable. On the right, parameters for the log-normal distribution are
fit separately. The fits are slightly different from XXX

\begin{Schunk}
\begin{figure}

{\centering \includegraphics[width=\textwidth]{qqplotr_files/figure-latex/weights-log-1} 

}

\caption[Sample (200 men and 200 women) of weights]{Sample (200 men and 200 women) of weights. On the left, the theoretical distribution  is changed to a log normal. On the right, we additionally estimate shift and scale parameters for each of the genders separately before comparing distributions to a log-normal.}\label{fig:weights-log}
\end{figure}
\end{Schunk}

\begin{table}

\caption{\label{tab:table}this table is just for us at the moment}
\centering
\begin{tabular}[t]{r|r|r|r|r}
\hline
SEX & mean\_wt & mean\_log\_wt & sd\_wt & sd\_log\_wt\\
\hline
1 & 91.50342 & 4.495766 & 19.22981 & 0.2011216\\
\hline
2 & 74.37085 & 4.282780 & 17.90126 & 0.2254663\\
\hline
\end{tabular}
\end{table}

\subsubsection{Using a other
distributions}\label{using-a-other-distributions}

Using the capabilities of \pkg{qqplotr} with the distributions
implemented in the \pkg{stats} package is relatively straightfoward,
since the implementation allows you to specify the suffix
(i.e.~distribution and or abbreviation) via the \texttt{distribution}
argument and the parameter estimates via \texttt{dparams} argument.
However, there are times when the distributions in \pkg{stats} are not
sufficient for the demands of the analysis. For example, there is no
left-skewed distribution listed. User-coded distributions or
distributions from other packages can be used with \pkg{qqplotr} as long
as the distributions are defined following the conventions laid out in
the \pkg{stats} package. Specfically, for some distribution there must
be density/mass (\texttt{d} prefix), CDF (\texttt{p} prefix), quantile
(\texttt{q} prefix), and simulation (\texttt{r} prefix) functions. In
this section we illustrate the use of the smallest extreme value
distribution (SEV).

To qualify for the Olympics in the men's long jump in 2012, athletes had
to either meet/exceed the 8.1 meter standard or place in the top twelve.
During the qualification events, each athlete was able to jump three
times, and their best (i.e.~longest) jump is treated as the result.
Figure \ref{fig:jump-density} shows a density plot of the results, which
are cleaerly left skewed.

\begin{Schunk}
\begin{figure}

{\centering \includegraphics[width=0.45\textwidth]{qqplotr_files/figure-latex/jump-density-1} 

}

\caption[Density plot of the 2012 men's long jump qualifying round]{Density plot of the 2012 men's long jump qualifying round. The distances are clearly left skewed.}\label{fig:jump-density}
\end{figure}
\end{Schunk}

In order to model the jump distances we must first define a left-skewed
distribution. Below, we define the suite of distribution functions
necessary to utilize the SEV distribution.

\begin{Schunk}
\begin{Sinput}
# CDF
psev <- function(q, mu = 0, sigma = 1) {
    z <- (q - mu) / sigma
    1 - exp(-exp(z))
}

# PDF
dsev <- function(x,  mu = 0, sigma = 1) {
  z <- (x - mu) / sigma
  (1 / sigma) * exp(z - exp(z))
}

# Quantile function
qsev <- function(p,  mu = 0, sigma = 1) {
  mu + log(-log(1 - p)) * sigma
}

# Simulation function
rsev <- function(n,  mu = 0, sigma = 1) {
  qsev(runif(n), mu, sigma)
}
\end{Sinput}
\end{Schunk}

With the \texttt{*sev} distribution functions in hand, we can create a
Q-Q plot to assess the appropriateness of the SEV model (Figure
\ref{fig:sev-qq}). The Q-Q plot show that the distances do not
substantially deviate from the SEV model, so we have found an adequate
representation of the distances.

\begin{Schunk}
\begin{Sinput}
ggplot(longjump, aes(sample = distance)) +
  stat_qq_band(distribution = "sev", dparams=list(mu=0, sigma=1), alpha = 0.3) +
  stat_qq_line(distribution = "sev", dparams=list(mu=0, sigma=1)) +
  stat_qq_point(distribution = "sev", dparams=list(mu=0, sigma=1)) +
  xlab("Theoretical quantiles") +
  ylab("Jump distance (in m)") +
  theme_bw()
\end{Sinput}
\begin{figure}

{\centering \includegraphics[width=0.45\textwidth]{qqplotr_files/figure-latex/sev-qq-1} 

}

\caption[Q-Q plot comparing the long jump distances to the standard SEV distribution]{Q-Q plot comparing the long jump distances to the standard SEV distribution. The SEV distribution appear to adequately model the distances.}\label{fig:sev-qq}
\end{figure}
\end{Schunk}

\subsubsection{Detrending Q-Q plots}\label{detrending-q-q-plots}

\pkg{qqplotr} also allows for Q-Q plots to be \emph{detrended}. In a
detrended Q-Q plot, the \(y\)-axis shows the difference between the
empirical quantile and the reference line (i.e.~the theoretical
distribution). This layout directly plots what we want viewers to
assess---the difference between the distributions being compared---which
\citet{Loy2016-fg} found to be more powerful than other designs, so long
at the \(y\)-axis limits are set so that the aspect ratio is kept the
same as in the traditional Q-Q plot.

For example, Figure \ref{fig:detrend-sev} compares the standard Q-Q plot
shown in Figure \ref{fig:sev-qq} with a detrended version by adding the
argument \texttt{detrend\ =\ TRUE} to the \texttt{stat\_qq\_band},
\texttt{stat\_qq\_line}, and \texttt{stat\_qq\_point} calls.

\begin{Schunk}
\begin{figure}

{\centering \includegraphics[width=\textwidth]{qqplotr_files/figure-latex/detrend-sev-1} 

}

\caption[Q-Q plots assessing the appropriateness of the SEV distribution for the long jump data]{Q-Q plots assessing the appropriateness of the SEV distribution for the long jump data. On the left, a standard Q-Q plot is shown. On the right, we detrend the Q-Q plot by plotting the differences between the empirical quantiles and reference line on the $y$-axis.}\label{fig:detrend-sev}
\end{figure}
\end{Schunk}

\begin{Schunk}
\begin{Sinput}
ggplot(longjump, aes(sample = distance)) +
  stat_qq_band(distribution = "sev", alpha = 0.3, 
               detrend = TRUE, dparams=c(mu=0, sigma=1)) +
  stat_qq_line(distribution = "sev", 
               detrend = TRUE, dparams=c(mu=0, sigma=1)) +
  stat_qq_point(distribution = "sev", 
                detrend = TRUE, dparams=c(mu=0, sigma=1)) +
  xlab("Theoretical quantiles") +
  ylab("Differences") +
  theme_bw() +
  theme(aspect.ratio = 1)
\end{Sinput}
\end{Schunk}

In order to create a so-called \emph{adjusted} detrended Q-Q plot
\citep{Loy2016-fg} the aspect ratio must also be set to 1. If the aspect
ratio is not adjusted in this way, an \emph{ordinary} detrended Q-Q plot
is created, which is known to lhave lower power than the standard Q-Q
plot in some situations \citep{Loy2016-fg}.

\subsection{Discussion}\label{discussion}

This paper presents the \pkg{qqplotr} package, an extension of
\pkg{ggplot2} that implements Q-Q plots in both the standard and
de-trended orientations, reference lines for Q-Q plots, and confidence
bands for Q-Q plots. The examples illustrate how to create Q-Q plots for
non-standard distributions found outside of the \pkg{stats}, de-trend
Q-Q plots, and create Q-Q plots when data are grouped. Further, in the
BRFSS example, we illustrated how jittering can be used in Q-Q plots to
better compare discretized data to a continuous distribution.

Q-Q plots are member of the larger probability plotting family, and we
are working to incorporate additional plots into \pkg{qqplotr}\ldots{}.
NEED TO DISUCSS P-P PLOTS SOMEHOW

XXX P-P plots here Write this section once the rest of the paper is
done.

\begin{Schunk}
\begin{figure}

{\centering \includegraphics[width=.31\linewidth]{qqplotr_files/figure-latex/unnamed-chunk-4-1} 

}

\caption{\label{fig:pp-designs}Illustrating different designs of probability plots.}\label{fig:unnamed-chunk-4}
\end{figure}
\end{Schunk}

\bibliography{RJreferences}

\subsection{Acknowledgements}\label{acknowledgements}

This work was partially funded by Google Summer of Code 2017.

\address{%
Alexandre Almeida\\
University of Campinas\\
Institute of Computing\\ Campinas, Brazil 13083-852\\
}
\href{mailto:almeida.xan@gmail.com}{\nolinkurl{almeida.xan@gmail.com}}

\address{%
Adam Loy\\
Carleton College\\
Department of Mathematics and Statistics\\ Northfield, MN 55057\\
}
\href{mailto:aloy@carleton.edu}{\nolinkurl{aloy@carleton.edu}}

\address{%
Heike Hofmann\\
Iowa State University\\
Department of Statistics\\ Ames, IA 50011-1210\\
}
\href{mailto:hofmann@iastate.edu}{\nolinkurl{hofmann@iastate.edu}}

