% !TeX root = RJwrapper.tex
\title{Capitalized Title Here}
\author{by Author One, Author Two, Author Three}

\maketitle

\abstract{%
An abstract of less than 150 words.
}

\subsection{TODO:}\label{todo}

\begin{itemize}
\tightlist
\item
  Determine order of authorship
\item
  Review Q-Q plots and P-P plots, including other arrangements, and what
  is implemented in other packages
\item
  Write about what the package implements
\item
  Give examples

  \begin{itemize}
  \tightlist
  \item
    Heike: BRFSS example
  \end{itemize}
\item
  Intro/conclusion
\item
  Abstract
\end{itemize}

\subsection{Introduction}\label{introduction}

Univariate distributional assessment is a common thread throughout
statistical analyses during both the exploratory and confirmatory
stages. When we begin exploring a new data set we often consider the
distribution of individual variables before moving on to explore
multivariate relationships. After a model has been fit to a data set, we
must assess whether the distributional assumptions made were reasonable,
and if they are not we then must understand the impact this has on the
conclusions. Graphics provide are arguably the most common way to carry
out these univariate assessments. While there are many graphics that can
be used for distribution exploration and assessment, probability
plotting is one of the most common graphical approaches used.

\begin{Schunk}
\begin{figure}

{\centering \includegraphics[width=.49\linewidth]{qqplotr_files/figure-latex/unnamed-chunk-1-1} \includegraphics[width=.49\linewidth]{qqplotr_files/figure-latex/unnamed-chunk-1-2} 

}

\caption{\label{fig:cdfs}Illustrating what quantities are being plotted for Q-Q and P-P plots.}\label{fig:unnamed-chunk-1}
\end{figure}
\end{Schunk}

Probability plotting refers to a family of methods based on the
cumulative distribution function (CDF), most notably quantile (Q-Q)
plots and probability (P-P) plots \citep{Wilk1968-ii}. Figure
\ref{fig:cdfs} displays two CDFs to illustrate the definition of Q-Q and
P-P plots. Let \(F_x(q)\) and \(F_y(q)\) denote the two CDFs, and let
\(q_x(p) = F_x^{-1}(p)\) and \(q_y(p) = F_y^{-1}(p)\) denote the
quantile functions for each distribution. A Q-Q plot is constructed by
plotting the \(q_y(p)\) against \(q_x(p)\) for various \(p\). A P-P plot
is constructed by plotting \(F_x(q)\) against \(F_y(q)\) for various
quantiles, \(q\). In practice, the distribution of a sample is often
compared to a theoretical distribution. Here, the empirical CDF is used
in place of \(F_y\). Regardless of the plot constructed, if the two
distributions are the same, then the scattplots will be linear with
slope 1 and intercept 0. Additionally, Q-Q plots are invariant to linear
transformations, so if two random variables differ by a linear
transformation a Q-Q plot showing draws from their distributions will
still be linear, but with a different slope and intercept. P-P plots are
sensitive to linear transformations.

While the basic form of both the Q-Q and P-P plots is a scatterplot,
additional graphical elements are often added to aid in distributional
assessment. For Q-Q plots, a reference line is often drawn through the
points \((F_x(.25), F_y(.25))\) and \((F_x(.75), F_y(.75))\). For P-P
plots a reference line with slope 1 and intercept 0 is used. In both
plots, pointwise or simultaneous confidence bands are often added around
the reference line to further aid in the visual assessment.

Innovations to Q-Q and P-P plots have also been proposed.
\citet{Loy2016-fg} discuss the creation of detrended Q-Q plots, where
the \(y\)-axis is changed to show the difference \(q_y - q_x\).
Consequently, the line representing the agreement between the
distribution is the \(x\)-axis. \citet{Loy2016-fg} find that detrended
Q-Q plots are more powerful than other designs, so long at the
\(y\)-axis limits are set so that the aspect ratio is kept the same as
in the traditional Q-Q plot. In reliability and survival analysis,
probability plots often refer to a hybrid probability plot, there the
CDF of the proposed theoretical distribution is plotted against the
empirical order statistics, and transformations are applied to each axis
to linearize the CDF \citep[cf.][chapter 6]{Meeker1998}. This hybrid
probability plot is invariant to linear transformations.

\textbf{ADD EXAMPLES OF THE Q-Q PLOT CONFIGURATIONS FOR CLARITY}

Q-Q plots have been implemented in various forms in R, but none provide
a complete implementation of the probability plotting framework. Normal
quantile plots, where a sample is compared to the standard normal
distribution, are implemented using the \texttt{qqplot} and
\texttt{qqline} in \pkg{base} graphics \citep{R}. \texttt{qqmath} in
\pkg{lattice} provides a general framework for Q-Q plots, comparing a
sample to any theoretical distribution by specifying the quantile
function \citep{lattice}. \texttt{qqPlot} in the \pkg{car} package also
allows for the assessment of non-normal distribution and adds pointwise
confidence bands based on the standard errors of the order statistics or
the parametric bootstrap \citep{car}. \pkg{ggplot2} provides
\texttt{geom\_qq} and \texttt{geom\_qq\_line}, enabling the creation of
traditional Q-Q plots with a reference line, much like those created
using \texttt{qqmath}. \pkg{qqplotr} extends
\textbackslash{}pkg\{ggplot2 to provide the most complete implementation
of probability plotting.

In the remainder of this paper, we introduce the probability plotting
framework provided by \pkg{qqplotr}\ldots{} FILL THIS IN ONCE OTHER
SECTIONS ARE WRITTEN\ldots{}

TODO: FIGURE OUT WHERE TO INTRODUCE TS BANDS \citep{Aldor-Noiman2013-xw}

\subsection{\texorpdfstring{Implementing probability plots in the
\pkg{ggplot2}
framework}{Implementing probability plots in the  framework}}\label{implementing-probability-plots-in-the-framework}

\subsection{Examples}\label{examples}

In this section, we demonstrate the capabilities of the \textbf{qqplotr}
package.

\begin{Schunk}
\begin{Sinput}
library(qqplotr)
\end{Sinput}
\end{Schunk}

\subsection{Summary}\label{summary}

This file is only a basic article template. For full details of
\emph{The R Journal} style and information on how to prepare your
article for submission, see the
\href{https://journal.r-project.org/share/author-guide.pdf}{Instructions
for Authors}. \bibliography{RJreferences}

\subsection{Acknowledgements}\label{acknowledgements}

Mention GSoC here\ldots{}

\address{%
Author One\\
Affiliation\\
line 1\\ line 2\\
}
\href{mailto:author1@work}{\nolinkurl{author1@work}}

\address{%
Author Two\\
Affiliation\\
line 1\\ line 2\\
}
\href{mailto:author2@work}{\nolinkurl{author2@work}}

\address{%
Author Three\\
Affiliation\\
line 1\\ line 2\\
}
\href{mailto:author3@work}{\nolinkurl{author3@work}}

