% !TeX root = RJwrapper.tex
\title{Capitalized Title Here}
\author{by Author One, Author Two, Author Three}

\maketitle

\abstract{%
An abstract of less than 150 words.
}

\subsection{TODO:}\label{todo}

\begin{itemize}
\tightlist
\item
  Review Q-Q plots and P-P plots, including other arrangements, and what
  is implemented in other packages
\item
  Write about what the package implements
\item
  Give examples

  \begin{itemize}
  \tightlist
  \item
    Heike: BRFSS example
  \end{itemize}
\item
  Intro/conclusion
\item
  Abstract
\end{itemize}

\subsection{Introduction}\label{introduction}

From GSoC proposal:

Quantile-quantile (Q-Q) plots are a powerful ways of visually diagnosing
distributional assumptions of a random variable. Help with this
assessment is provided by a line through points in the first and third
quartiles of the empirical and theoretical distributions (commonly known
as qqline) as well as by a confidence band or pointwise intervals around
the line. It has been shown by Aldor-Noiman et al (2013) and Loy et al
(2016) that both the choice of the interval around the line and the
design of the Q-Q plot, such as a rotation by 90 degree, have an impact
on our ability to use Q-Q plots. In the ggplot2 framework (Wickham, 2009
and 2016) quantile-quantile plots are supported by the stat\_qq and the
geom geom\_qq, which is connected to drawing the points for the
quantile-quantile- plot. We are proposing to add extensions to the
ggplot2 framework for adding a Q-Q line as well as support for bands
around this line. Since ggplot2 version 2.0.0 the way that geoms are
support has been completely overhauled, which makes extensions much
easier to write.

Q-Q plots have been implemented in various forms in R, starting with
qqplot and qqline in the base package. However, the functionality within
the ggplot2 package is restricted to stat\_qq and geom\_qq, both of
which are only concerned with the placement of points in a Q-Q plot. By
providing functionality for the drawing of the Q-Q line and a confidence
region in form of a geom additional ggplot2 tools such as facetting and
layering are made available to the analyst.

References to incorporate:

\citet{Wilk1968-ii} for general Q-Q and P-P plot reference

\citet{Aldor-Noiman2013-xw} for TS bands

\citet{Loy2016-fg} for detrended Q-Q plots, etc.

Previously implemented Q-Q or P-P plots:

\begin{itemize}
\tightlist
\item
  \texttt{qqnorm} and \texttt{qqline} functions in \textbf{base}
  \citep{R}
\item
  \texttt{qqmath} in \textbf{lattice} \citep{lattice}
\item
  \texttt{qqPlot} in \textbf{car} \citep{car}
\item
  \texttt{probplot} in \textbf{e1071} \citep{e1071}
\item
  \texttt{geom\_qq} and \texttt{geom\_qq\_line} in \textbf{ggplot2}
  \citep{ggplot2}
\end{itemize}

\subsection{Section title in sentence
case}\label{section-title-in-sentence-case}

\subsection{Examples}\label{examples}

In this section, we demonstrate the capabilities of the \textbf{qqplotr}
package.

\begin{Schunk}
\begin{Sinput}
library(qqplotr)
\end{Sinput}
\end{Schunk}

\subsection{Summary}\label{summary}

This file is only a basic article template. For full details of
\emph{The R Journal} style and information on how to prepare your
article for submission, see the
\href{https://journal.r-project.org/share/author-guide.pdf}{Instructions
for Authors}. \bibliography{RJreferences}

\subsection{Acknowledgements}\label{acknowledgements}

Mention GSoC here\ldots{}

\address{%
Author One\\
Affiliation\\
line 1\\ line 2\\
}
\href{mailto:author1@work}{\nolinkurl{author1@work}}

\address{%
Author Two\\
Affiliation\\
line 1\\ line 2\\
}
\href{mailto:author2@work}{\nolinkurl{author2@work}}

\address{%
Author Three\\
Affiliation\\
line 1\\ line 2\\
}
\href{mailto:author3@work}{\nolinkurl{author3@work}}

