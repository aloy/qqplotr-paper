% !TeX root = RJwrapper.tex
\title{Capitalized Title Here}
\author{by Author One, Author Two, Author Three}

\maketitle

\abstract{%
An abstract of less than 150 words.
}

\subsection{TODO:}\label{todo}

\begin{itemize}
\tightlist
\item
  Order of authorship?
\item
  Review Q-Q plots and P-P plots, including other arrangements, and what
  is implemented in other packages
\item
  Write about what the package implements
\item
  Give examples

  \begin{itemize}
  \tightlist
  \item
    Heike: BRFSS example
  \end{itemize}
\item
  Intro/conclusion
\item
  Abstract
\end{itemize}

\subsection{Introduction}\label{introduction}

Univariate distributional assessment is a common thread throughout
statistical analyses during both the exploratory and confirmatory
stages. When we begin exploring a new data set we often consider the
distribution of individual variables before moving on to explore
multivariate relationships. After a model has been fit to a data set, we
must assess whether the distributional assumptions made were reasonable,
and if they are not we then must understand the impact this has on the
conclusions. Graphics provide arguably the most common way to carry out
these univariate assessments. While there are many graphical methods
that can be used for distribution exploration and assessment,
probability plotting is one of the most common graphical approaches
used.

Probability plotting refers to a family of methods based on the
cumulative distribution function (CDF), most notably quantile (Q-Q)
plots and probability (P-P) plots \citep{Wilk1968-ii}. In this paper, we
focus on comparing an empirical distribution to a theoretical
distribution. Let \(Y_1, \ldots, Y_n\) denote a random sample from an
unknown population, and let \(\widehat{F}_y(q)\) be the empirical
cumulative distribution obtained from the sample. Further, let \(F(q)\)
denote the CDF of a proposed dsitribution for the sample. A Q-Q plot is
constructed by plotting the quantiles of the empirical distribution,
\(q_y(p) = F_y^{-1}(p)\), against the corresponding quantiles of the
theoretical distribution, \(q(p) = F^{-1}(p)\). This constuction is
illustrated in Figure \ref{fig:qq}. A P-P plot is constructed by
plotting \(F(q)\) against \(\widehat{F}_y(q)\) for various quantiles,
\(q\). This constuction is illustrated in Figure \ref{fig:pp}.
Regardless of the plot constructed, if the two distributions are
identical, then the scattplots will be linear with slope 1 and intercept
0. Additionally, Q-Q plots are invariant to linear transformations, so
if two random variables differ by a linear transformation a Q-Q plot
showing draws from their distributions will still be linear, but with a
different slope and intercept, as seen in Figure \ref{fig:qq}. P-P plots
are sensitive to linear transformations.

\begin{Schunk}
\begin{Soutput}
#> 
#> Attaching package: 'qqplotr'
\end{Soutput}
\begin{Soutput}
#> The following objects are masked from 'package:ggplot2':
#> 
#>     stat_qq_line, StatQqLine
\end{Soutput}
\begin{figure}

{\centering \includegraphics[width=.45\linewidth]{qqplotr_files/figure-latex/unnamed-chunk-1-1} \includegraphics[width=.45\linewidth]{qqplotr_files/figure-latex/unnamed-chunk-1-2} 

}

\caption{\label{fig:qq}Illustrating what quantities are being plotted for Q-Q plots.}\label{fig:unnamed-chunk-1}
\end{figure}
\end{Schunk}

\begin{Schunk}
\begin{figure}

{\centering \includegraphics[width=.45\linewidth]{qqplotr_files/figure-latex/unnamed-chunk-2-1} \includegraphics[width=.45\linewidth]{qqplotr_files/figure-latex/unnamed-chunk-2-2} 

}

\caption{\label{fig:pp}Illustrating what quantities are being plotted for P-P plots.}\label{fig:unnamed-chunk-2}
\end{figure}
\end{Schunk}

While the basic form of both the Q-Q and P-P plots is a scatterplot,
additional graphical elements are often added to aid in distributional
assessment. For Q-Q plots, a reference line is often drawn through the
points \((q(.25), q_y(.25))\) and \((q(.75), q_y(.75))\). For P-P plots
a reference line with slope 1 and intercept 0 is used. In both plots,
pointwise or simultaneous confidence bands are often added around the
reference line to further aid in the visual assessment.

Innovations to Q-Q and P-P plots have also been proposed.
\citet{Loy2016-fg} discuss the creation of detrended Q-Q plots, where
the \(y\)-axis is changed to show the difference betweeh \(q_y\) and the
reference line. Consequently, the line representing the agreement with
the theoretical distribution is the \(x\)-axis. \citet{Loy2016-fg} find
that detrended Q-Q plots are more powerful than other designs, so long
at the \(y\)-axis limits are set so that the aspect ratio is kept the
same as in the traditional Q-Q plot. In reliability and survival
analysis, probability plots often refer to a hybrid probability plot,
there the CDF of the proposed theoretical distribution is plotted
against the empirical order statistics, and transformations are applied
to each axis to linearize the CDF \citep[cf.][chapter 6]{Meeker1998}.
This hybrid probability plot is invariant to linear transformations.

\begin{Schunk}
\begin{figure}

{\centering \includegraphics[width=.31\linewidth]{qqplotr_files/figure-latex/unnamed-chunk-3-1} \includegraphics[width=.31\linewidth]{qqplotr_files/figure-latex/unnamed-chunk-3-2} \includegraphics[width=.31\linewidth]{qqplotr_files/figure-latex/unnamed-chunk-3-3} 

}

\caption{\label{fig:pp}Illustrating different designs of probability plots.}\label{fig:unnamed-chunk-3}
\end{figure}
\end{Schunk}

Q-Q plots have been implemented in various forms in R, but none provide
a complete implementation of the probability plotting framework. Normal
quantile plots, where a sample is compared to the standard normal
distribution, are implemented using the \texttt{qqplot} and
\texttt{qqline} in \pkg{base} graphics \citep{R}. \texttt{qqmath} in
\pkg{lattice} provides a general framework for Q-Q plots, comparing a
sample to any theoretical distribution by specifying the quantile
function \citep{lattice}. \texttt{qqPlot} in the \pkg{car} package also
allows for the assessment of non-normal distribution and adds pointwise
confidence bands based on the standard errors of the order statistics or
the parametric bootstrap \citep{car}. \pkg{ggplot2} provides
\texttt{geom\_qq} and \texttt{geom\_qq\_line}, enabling the creation of
traditional Q-Q plots with a reference line, much like those created
using \texttt{qqmath}. \pkg{qqplotr} extends
\textbackslash{}pkg\{ggplot2 to provide the most complete implementation
of probability plotting.

In the remainder of this paper, we introduce the probability plotting
framework provided by \pkg{qqplotr}\ldots{} FILL THIS IN ONCE OTHER
SECTIONS ARE WRITTEN\ldots{}

TODO: FIGURE OUT WHERE TO INTRODUCE TS BANDS \citep{Aldor-Noiman2013-xw}

\subsection{\texorpdfstring{Implementing probability plots in the
\pkg{ggplot2}
framework}{Implementing probability plots in the  framework}}\label{implementing-probability-plots-in-the-framework}

\subsubsection{Q-Q plots}\label{q-q-plots}

\subsubsection{P-P plots}\label{p-p-plots}

\subsection{Examples}\label{examples}

In this section, we demonstrate the capabilities of the \textbf{qqplotr}
package.

\begin{Schunk}
\begin{Sinput}
library(qqplotr)
\end{Sinput}
\end{Schunk}

\subsubsection{BRFSS example}\label{brfss-example}

The Center for Disease Control and Prevention runs an annual telephone
survey, the Behavioral Risk Factor Surveillance System (BRFSS), to keep
track of the US populations' `health-related risk behaviors, chronic
health conditions, and use of preventive services'.\\
Close to half a million interviews are conducted each year. Here, we are
focussing on the 2012 responses for Iowa. 7166 responses were gathered
across 359 questions and derived variables. Among these, are people's
height and weight, which we are going to assess using a normal
assumption:

\begin{Schunk}

\includegraphics{qqplotr_files/figure-latex/unnamed-chunk-5-1} \end{Schunk}

\subsection{Summary}\label{summary}

Write this section once the rest of the paper is done.

\bibliography{RJreferences}

\subsection{Acknowledgements}\label{acknowledgements}

Mention GSoC here\ldots{}

\address{%
Author One\\
Affiliation\\
line 1\\ line 2\\
}
\href{mailto:author1@work}{\nolinkurl{author1@work}}

\address{%
Author Two\\
Affiliation\\
line 1\\ line 2\\
}
\href{mailto:author2@work}{\nolinkurl{author2@work}}

\address{%
Author Three\\
Affiliation\\
line 1\\ line 2\\
}
\href{mailto:author3@work}{\nolinkurl{author3@work}}

